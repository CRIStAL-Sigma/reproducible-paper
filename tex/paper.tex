\documentclass{article}

\usepackage{amsfonts,amsmath,amssymb}
\usepackage{fontawesome} % For fancy icons
\usepackage[a4paper]{geometry}
\geometry{rmargin=4cm,vmargin=3cm}
\usepackage{graphicx}
\usepackage{hyperref}
\hypersetup{
  colorlinks=true,
  linkcolor=blue,
  urlcolor=gray,
  citecolor=blue
}
\usepackage[utf8x]{inputenc}
\usepackage{natbib}
\usepackage{xcolor}
% Specific to this .tex
\newcommand\HW{\text{Hello World!}}

\begin{document}

%%%%%%%%%%%
%% TITLE %%
%%%%%%%%%%%

\title{Reproducible paper}

\author{Guillaume Gautier}

\date{ Last compilation \today }

\maketitle

%%%%%%%%%%%%%%
%% Abstract %%
%%%%%%%%%%%%%%

\abstract{A collaborative paper/note with \LaTeX.
The typos are made intentionally \faSmileO.\\
The project is available on \href{https://github.com/CRIStAL-Sigma/reproducible-paper}{GitHub} \faGithub}

%%%%%%%%%%
%% BODY %%
%%%%%%%%%%

\section{Equations} % (fold)
\label{sec:equations}

	I was told that
	\begin{equation}
		\sum_{n=1}^{\infty} \frac{1}{n^2} = \frac{\pi}{6}.
	\end{equation}

% section equations (end)

\section{Images} % (fold)
\label{sec:images}

	\subsection{Basic} % (fold)
	\label{sub:basic}
		
		\begin{figure}[h]
			\label{fig:collabocats}
			\centering
			\includegraphics[scale=0.1]{images/collabocats.jpg}
			\caption{Use \href{https://git-scm.com/}{\textcolor{black}{\faGit}}\ and \href{https://github.com/}{\textcolor{black}{\faGithub}}\ to collaborate}
		\end{figure}

	% subsection basic (end)

	\subsection{with TikZ} % (fold)
	\label{sub:with_tikz}
		
	% subsection with_tikz (end)

% section images (end)

\section{Random} % (fold)
\label{sec:random}

	Take a look at the \href{https://en.wikipedia.org/wiki/Linus_Torvalds}{WikipediA} page of \citet{Python}, the father of Python.

% section random (end)

%%%%%%%%%%%%
%% BIBLIO %%
%%%%%%%%%%%%

\bibliographystyle{plainnat}
\bibliography{biblio}

\end{document}